\documentclass[xcolor=table]{beamer}

\usepackage{lscape, amsmath, amsfonts, amssymb, setspace, theorem, wrapfig, graphicx, float, multirow, subfig, color, rotating, multicol, datetime, natbib, venndiagram, pstricks, xkeyval, tikz, etoolbox, url, hyperref, nth}

\usepackage[T1]{fontenc}
\usepackage[latin1]{inputenc}
\usepackage[english]{babel}
\usetikzlibrary{arrows}

\title{GV217 - Conflict Analysis}
\subtitle{University of Essex - Department of Government}
\date{Week 22 -- 28 February, 2020}				% or you can specify a date, just write it down instead of "\today"
\author{Lorenzo Crippa} 

\usetheme[progressbar=frametitle]{metropolis}
\usecolortheme{seahorse}						% try others: wolverine; crane...

\begin{document}
\frame{
\titlepage
}

\section{Gender and conflicts}

\frame{
\frametitle{Wartime sexual violence}
Two orders of explanations for wartime sexual violence: \pause
\begin{enumerate}
\item Previous explanations \pause
	\begin{itemize}
	\item Ethnic war/secession -- humiliation, terror
	\item Genocide (\textbf{definition}) -- rape as part of genocide
	\item Gender inequality -- acceptance of violence against women
	\item State collapse (weakening social norms) -- opportunism
	\item Lootable resources -- attracts opportunists; reduced accountability of rebel groups \pause
	\end{itemize}
\item New research: \pause
	\begin{itemize}
	\item Combatant socialisation is needed when recruitment is forced
	\item Shared experience creates bonds of loyalty and esteem $=$ social cohesion
	\end{itemize}
\end{enumerate}
}

\frame{
\frametitle{Definition of genocide}
\textbf{The 1948 Genocide Convention:}

``Causing serious bodily or mental harm to members of the group and/or deliberately inflicting on the group conditions of life calculated to bring about its physical destruction in whole or in part.''
}

\frame{
\frametitle{Female participation to combat}
\textcolor{red}{How do you think women participate to combat for \textbf{leftist}, \textbf{secular}, \textbf{religious}, \textbf{Islamist}, or \textbf{nationalist} groups?} \pause

Women are: \pause
\begin{itemize}
\item \textbf{More likely} to participate in fighting for \textbf{leftist groups} \pause
\item \textbf{More likely} to participate in fighting for \textbf{secular groups} \pause
\item \textbf{Less likely} to participate in fighting for \textbf{religious groups} \pause
\item \textbf{Less likely} to participate in fighting for \textbf{Islamist groups} \pause
\item \textbf{No more or less likely} to participate in fighting for \textbf{nationalist groups}
\end{itemize} \pause

\textcolor{red}{How does recruitment by different groups work? Are women joining voluntarily or are they actively recruited? The two imply different mechanisms!}
}

\frame{
\frametitle{Example 1: Women's participation in combat -- PKK}
Let's watch \href{https://www.youtube.com/watch?v=fQZR6xzDkjc}{\textbf{this video}}. \pause

Questions to discuss:
\begin{enumerate}
\item Do you think it is because of social norms that we see less women fighting? Or are men naturally more likely to take part in violent conflict? \pause
\item \textbf{Why} did the PKK recruit women fighters? \pause
\item Would the same be true for IS? Why? 
\end{enumerate}
}

\frame{
\frametitle{Example 2: Women's mediation of combat -- Liberia}
2003: Women of Liberia Mass Action for Peace campaign during the Second Liberian Civil War. ``Liberia's Women in White'' \pause
\begin{itemize}
\item Liberian women pressured their men to pursue peace or lose physical intimacy with them \pause
\item They gathered in the capital Monrovia, sitting, praying, singing, dressed in all-white clothing \pause
\item They first obtained then-president Charles Taylor to agree to attend peace talks with leaders of the warring factions \pause
\item They performed ``corridor lobbying'', waiting for negotiators as they entered and exited meeting rooms during breaks. \pause
\item They blocked entry and exit point, stopping negotiators from leaving the talks without a resolution. \pause
\item They contributed to the signing of the 2003 peace agreement
\end{itemize}
}

\frame{
\frametitle{UN Resolution 1325}
\begin{itemize}
\item Conventionally, war has been thought of as a man's domain. \pause
	\begin{itemize}
	\item[--] How are women affected by war?
	\item[--] What are the roles of women in conflicts?
	\end{itemize} \pause
\item What challenges does resolution 1325 (2000) hope to address?\pause
	\begin{itemize}
	\item[--] Acknowledges impact of war and conflicts on women and girls and their rights \pause
	\item[--] Proposes a gender perspective to consider their special needs in wartimes \pause
	\item[--] Acknowledges and valorises their role as conflict prevention, resolution and peace-building through adopting a gender perspective on these themes
	\end{itemize}
\end{itemize}
}

\frame{
\frametitle{Conclusion}
\begin{center}
All clear? More questions? \\
See you next week!
\end{center}
}

\end{document}