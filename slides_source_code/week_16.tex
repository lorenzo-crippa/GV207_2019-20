\documentclass[xcolor=table]{beamer}

\usepackage{lscape, amsmath, amsfonts, amssymb, setspace, theorem, wrapfig, graphicx, float, multirow, subfig, color, rotating, multicol, datetime, natbib, venndiagram, pstricks, xkeyval, tikz, etoolbox, url, hyperref}

\usepackage[T1]{fontenc}
\usepackage[latin1]{inputenc}
\usepackage[english]{babel}

\title{GV217 - Conflict Analysis}
\subtitle{University of Essex - Department of Government}
\date{Week 16 -- 17 January, 2020}				% or you can specify a date, just write it down instead of "\today"
\author{Lorenzo Crippa} 

\usetheme[progressbar=frametitle]{metropolis}
\usecolortheme{seahorse}						% try others: wolverine; crane...

\begin{document}
\frame{
\titlepage
}

\frame{
\frametitle{Your GTA}

\begin{center}
Hi! My name is Lorenzo Crippa,

Email: l.crippa@essex.ac.uk

Office: 5B.153 (Department of Government)

Office hour: Monday 14:00 to 16:00
\end{center}
}

\frame{
\frametitle{This seminar}
When and where?
\begin{itemize}
\item Class is on Friday from 9 to 10am \pause
\item Room 4.204 (but in IT Lab N next week!) \pause
\end{itemize}

What are we doing together? \pause
\begin{itemize}
\item Go through concepts seen in class \pause
\item Apply concepts and theory to practical examples \pause
\item Work in groups and discuss
\end{itemize} \pause

When in doubt, read the syllabus!
}

\frame{
\frametitle{Ground rules}
Some (few) ground rules: \pause
\begin{itemize}
\item Prepare weekly readings! \pause
\item Raise hands to intervene \pause
\item There are no stupid questions. Always feel free to ask \pause
\item Respect each other's opinions
\end{itemize}
}

\frame{
\frametitle{Rules to decide}
Rules to decide together:
\begin{itemize}
\item Do you prefer to start at 9 and finish at 9:50am or to start at 9:10 and finish at 10am? 
\item Do you need slides? Is it ok if we use the seminar mailing list to disseminate them and communicate?
\end{itemize} \pause

Now introduce yourselves! What's your name? What do you study? What do you expect from this module?
}

\frame{
\frametitle{First assignment}
Fist assignment due in week 18
\begin{enumerate}
\item List the UCDP recorded conflicts in \textbf{2018}
\item Describe the conflicts: What patterns do you see among them?
	\begin{itemize}
	\item Readings and definitions of this week
	\item UCDP data (discussed next week)
	\end{itemize}
\end{enumerate} \pause
Useful resources from Uppsala Conflict Data Program (UCDP):

UCDP \textbf{definitions}: \href{https://www.pcr.uu.se/research/ucdp/definitions/}{click here} \\ UCDP \textbf{data}: \href{https://ucdp.uu.se/exploratory}{click here}
}

\frame{
\frametitle{Definition of \underline{state-based} armed conflict}
\textbf{Definition}: An armed conflict is a contested incompatibility that concerns a government and/or territory where the use of armed force between two parties, of which at least one is the government of a state, results in at least 25 battle-related deaths in one calendar year.
}

\frame{
\frametitle{Underlying concepts}
\begin{itemize}
\item \textbf{Incompatibility}: disagreement, where the demands of two parties cannot be met by the same resources at the same time \pause
	\begin{itemize}
	\item[--] Territory: Both parties claim the same piece of territory
	\item[--] Government: Both parties claim the right to rule over the same country \pause
	\end{itemize}
\item \textbf{Armed Force}: use of arms in organised violence \pause
\item \textbf{Government}: the party controlling the capital of the state \pause
\end{itemize}

Battle-related deaths occur in warfare \pause (battlefield fighting, bombardments or guerilla activities involving the armed forces of the warring parties). \pause All deaths -- military as well as civilian -- incurred during conflict are counted as battle-related deaths.
}

\frame{
\frametitle{Intrastate \emph{vs} interstate conflict}
\begin{itemize}
\item \textbf{Interstate}: Conflict between two or more governments \pause
\item \textbf{Intrastate}: Conflict between a government and a non-governmental party \pause
\item \textbf{Extrastate armed conflict}: Conflict between a state and non-state group outside its own territory \pause
\item \textbf{Intrastate with foreign involvement -- internationalized intrastate}:  An armed conflict between a government and a non-government party where the government side, the opposing side, or both sides, receive troop support from other governments that actively participate in the conflict.
\end{itemize} \pause

Can you provide examples of them?
}

\frame{
\frametitle{Exercise}
\begin{itemize}
\item Get in groups of 2-3 people
\item Try to identify which of the examples are armed conflict
\item Present your findings to another group
\item Do you agree? 
\end{itemize}

An armed conflict is a \textbf{contested incompatibility} that concerns \textbf{government and/or territory} where the \textbf{use of armed force between two parties}, of which \emph{at least one} is the \textbf{government of a state}, results in at least \textbf{25 battle-related deaths} in one calendar year.
}

\frame{
\frametitle{What is not armed conflict?}
Can you provide examples of what we would \emph{not} consider as armed conflict? \pause
\begin{itemize}
\item Terrorism \pause
\item Non-violent campaign \pause
\item Communal violence \pause
\item Genocide
\end{itemize}
}

\frame{
\frametitle{A real conflict or banditry in disguise?}
Questions and issues to consider:
\begin{itemize}
\item Why do we count deaths?
\item Should the definition include civilian deaths, and deaths of combatants outside of conflict, such as disease, in prison, etc.?
\item We consider calendar years
\item ``a politically significant event''
\end{itemize}
}

\frame{
\frametitle{Further resources to study conflicts}
\begin{itemize}
\item Research Institutions: e.g., UCDP, ACLED \pause
\item Think Tanks: e.g., Crisis Group, Chatham House, Carnegie, Council on Foreign Relations \pause
\item NGOs: e.g., Amnesty International, International Red Cross \pause
\item International news outlets: e.g., BBC, NYT, Aljazeera \pause
\end{itemize}

Include as many resources as possible to prevent bias! 
}

\frame{
\frametitle{Conclusion}
\begin{center}
All clear? More questions? \\
Thanks and see you next week!
\end{center}
}


\end{document}
